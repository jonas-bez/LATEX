\documentclass{article}
\usepackage{indentfirst}
\usepackage{graphicx}
\usepackage{amsmath}
\usepackage[utf8]{inputenc}

\title{Lista 2 Eletromagnetismo}
\author{Jonas Bezerra}
\date{March 2019}

\begin{document}

\maketitle

\section{Resnick}

\textbf{1}

$F=\frac{1}{4\pi \epsilon _0}. \frac{q.(Q-q)}{r^2}$

o valor que minimiza a função f(q)=q.(Q-q) temoqs que 

$Qq-q^2 \rightarrow \frac{d}{dq}.(Qq-q^2) \rightarrow Q-2q=0 \rightarrow q=\frac{Q}{2}\rightarrow \frac{q}{Q}=0,5 $\newline

\textbf{3}

$F=\frac{1}{4\pi \epsilon _0}. \frac{|q_1| |q_2|}{r^2}$\newline

$r^2=\frac{1}{4\pi \epsilon _0}. \frac{|q_1| |q_2|}{F} $\newline

$r=\sqrt{ \frac{1}{4\pi \epsilon _0}. \frac{|q_1| |q_2|}{F}}$\newline

$r=\sqrt{\frac{(8,9.10^{10}).(26,0.10^{-6}).(47,0.10^{-6})}{5,7}} = 1,39 m$\newline

\textbf{4}

$q= it$\newline

$q=carga(c)$\newline

$i=corrente(a)$\newline

$t=tempo(s)$\newline

$(2,5.10^{4}A).(20.10^{-6}s)=0,5 C$\newline

\textbf{5}

$F=\frac{1}{4\pi \epsilon _0}. \frac{|q_1| |q_2|}{r^2}$\newline

$q_1 = 3.10^{-6}$\newline

$q_2 = -1,5$\newline

$r = 12 cm \rightarrow 0,12m$\newline

$F=\frac{(8.99.10^{9}).(3.10^{-3}).(-1,5^{-6})}{(0,12)^2}$\newline

$F=\frac{40,455.10^{-3} }{0,0144} \rightarrow F=2,18 N$\newline

\textbf{6}

a) a massa da segunda partícula 

$m1a1 = m2a2$\newline

$(6,3.10^{-7}).(7)=m2.(9)$\newline

$m2=\frac{(6,3.10^{-7}).(7)}{9}$\newline

$m2 = 4,9.10^{-7}$\newline

b) O móduli da carga das partículas


$F=\frac{1}{4\pi \epsilon _0}. \frac{|q_1| |q_2|}{r^2}= m_1a_1$\newline

$(8.99.10^{9}). \frac{q^2}{(0,0032)^2}=(6,3.10^{-7}).(7)$\newline

$q=\sqrt{\frac{(6,3.10^{-7}).(7).(0,0032)^2}{(8.99.10^{9})}}$\newline

$q=7,08.10^{-11}$\newline

\textbf{7}\newline

Qual o valor da fração $\frac{q_1}{q_2}$\newline

$F=f_1_3 + f_2_3$\newline

$F=\frac{1}{4\pi\epsilon_0}.\frac{|q_1|.|q_3|}{(l_1_2 + l_2_3)^2}+\frac{1}{4\pi\epsilon_0}.\frac{|q2|.|q3|}{(l_2_3)^2}$\newline

Fazendo $l_1_2 = l_2_3$ Cada espaço = 1\newline

$\frac{1}{4\pi\epsilon_0}.\frac{|q_1|.|q_3|}{(l_1_2 + l_2_3)^2} = \frac{-1}{4\pi\epsilon_0}.\frac{|q_2|.|q_3|}{|l_2_3|}\rightarrow \frac{q1}{4} = \frac{-q2}{1} \rightarrow \frac{q1}{q2} = -4$\newline

\textbf{8}\newline

1º)\newline

C entra em contato com A e a carga de cada uma é 2Q.\newline C entra em contato com $B\rightarrow(2Q - 6Q = -4Q)$ dividida para as duas que da -2Q para cada.\newline

$F_1 = \frac{1}{4\pi\epsilon_0}.\frac{(2Q).(-2Q)}{d^2}$\newline

2º)\newline

C entra em  contato com B que deixa cada uma com -3Q\newline
C entra em contato com $A(-3Q+4Q = Q)\rightarrow\frac{Q}{2}$ para cada\newline

$F_2 = \frac{1}{4\pi\epsilon_0}. \frac{\frac{Q}{2}.(-3Q)}{d^2}$\newline

$\frac{F_2}{F_1} = \frac{\frac{3}{2}}{4} = 0,375$\newline

\textbf{10}\newline

$q_1 =q+4 = Q$ e $q_2 = q_3$\newline

a)$\frac{Q}{q}$ seja nula\newline

$F_1 = \frac{1}{4\pi\epsilon_0}.(\frac{Q.Q}{(a\sqrt{2})^2}.\cos45 + \frac{(|q_1|).(a)}{a^2}) = \frac{|q|^2}{4\pi\epsilon a^2}(\frac{\frac{-Q}{|q|}}{2\sqrt{2}} + 1)$\newline

Fazendo $F_1 = 0 \rightarrow \frac{Q}{|q|} = -2\sqrt{2}\rightarrow -2,83$\newline

b)Por simetria os valores são iguais\newline

$F_2 = \frac{1}{4\pi\epsilon_0}.(\frac{|q|^2}{(\sqrt{2}a)^2}.\sen45 - \frac{(|q|)(Q)}{a^2}) \rightarrow \frac{|q|^2}{4\pi\epsilon a^2}.(\frac{1}{2\sqrt{2}} - \frac{Q}{|q|})$\newline

$\frac{Q}{|q|} = \frac{-1}{2}.\sqrt{2} = 0,35$\newline

Como o valor de $F_1$ é diferente de $F_2$ não existe um valor para qual todos as partículas sejam nulas.\newline

\textbf{11}\newline

$Q_1 = -Q_2 = 100 nc$\newline
$Q_3 = -Q_4 = 200 nc$\newline

a)\newline
a componente X da força que a partícula 3 está submetida é\newline

$F_3_1 = \frac{|q3|}{4\pi\epsilon a^2}. (\frac{|q2|}{2\sqrt{2}} + |q_1|) = 8,99.10^9. \frac{2(1.10^{-7}}{(0,05)^2}. (\frac{1}{2\sqrt{2}} + 2) = 0,17 N$\newline

b)\newline
a força de Y é\newline

$F_3y = \frac{|q_3|}{4\pi\epsilon_0 a^2}. (-|q1| + \frac{|q2|}{2\sqrt{2}}) = 8,99.10^9.2\frac{1.10^{-7}}{(0,05)^2}.(-1 + \frac{1}{2\sqrt{2}}) = -0,046N$

\textbf{14}\newline

a)chamar de Q a carga 3\newline

$\frac{1}{4\pi \epsilon_0}.\frac{|q_1|Q}{(-a\frac{-a}{2})^2}=
\frac{1}{4\pi \epsilon_0}.\frac{|q_2|Q}{(a\frac{-a}{2})^2}$\newline

$|q_1|=9,0|q_2| \rightarrow \frac{q_1}{q_2}=9,0$\newline

b) nesse caso temos;\newline

$\frac{1}{4\pi \epsilon_0}.\frac{|q_1|Q}{(a\frac{-3a}{2})^2}=
\frac{1}{4\pi \epsilon_0}.\frac{|q_2|Q}{(a\frac{-3a}{2})^2}$

$|q_1|=25|q_2| \rightarrow \frac{q_1}{q_2}=-25$\newline



\section{Moysés}

\textbf{2.1}\newline

Força eletrostática = $\frac{1}{4\pi\epsilon_0}.\frac{e^2}{d^2}$\newline

Força gravitacional = $\frac{G . Me . Mp}{d^2}$\newline

dividindo $|Fe||Fg|$\newline


$\frac{e^2}{4\pi\epsilon_0 . 2}.\frac{d^2}{G.Me.Mp} \rightarrow$ Independe da dinâmica.\newline

$\frac{1}{4\pi\epsilon_0} = 9.10^9$

$Me = 99.10^{-31}Kg$\newline
$Mp = 1,67.10^{-27}Kg$\newline
$C = 1,60 . 10^{-19}$\newline
$g = 6,67.10^{-11}$\newline

\textbf{2.2}\newline

1 mol $\rightarrow$ 22,4l em NTP\newline
então em 1 litro remos $\frac{1}{22,4}mol$\newline

$\frac{1}{22,4}.6,023.10^23 = 2,6884.10^21$\newline

Como cada molécula tem 2 átomos\newline
Tem-se $5,3768.10^21$. Carga do elétron em coulombs\newline

Tem-se $8,6.10c$\newline

\textbf{2.5}\newline

\includegraphics[width=6cm]{le1.jpeg}\newline
a)\newline

I $t.sen\theta - F = 0$ em X\newline

II $t.cos\theta - mg = 0$\newline
(mg é o peso da carga)\newline

I = $t.sen\theta = F \rightarrow t = \frac{F}{sen\theta}$ - Substituindo em II.\newline

$\frac{F}{sen\theta}.cos\theta - mg = 0$ Onde $F = \frac{1}{4\pi\epsilon_0}.\frac{q.q}{(2(l.sen\theta))^2}$\newline

Substituindo F\newline

$(\frac{1}{4.\pi\epsilon_0}.\frac{q^2}{(2.l.sen\theta)^2}. sen\theta).cos\theta - Mg = 0$\newline

$Q^2.cos\theta = 16\pi\epsilon_0.t^2.Mg.sen\theta$\newline

b)\newline

$Q^2 cos\theta = 16\pi\epsilon_0.l^2.Mg.sen^3\theta$\newline

$Q^2 \cos30 = 16\pi\epsilon_0.(20)^2. Mg.sen^3(30)$\newline

$Q^2.\frac{\sqrt{3}}{2} = 4. 9.10^9.400^200.Mg.\frac{1}{8}$\newline

$Q^2 = 3600 - \sqrt{3}.Mg$\newline

$Q = 60 - \sqrt{3}. Mg$\newline

\textbf{2.6}

\includegraphics[width=8cm]{le2.jpeg}\newline

adotando $k= \frac{1}{4\pi \epsilon _0}$\newline

$F_1=K. \frac{|Q| |q|}{d^2} \hat r . \frac{d}{d} \rightarrow
\Vec{r} =|d|.\hat r \rightarrow (\frac{h \hat j}{3}-\frac{a \hat i}{2})$\newline

$(\frac{h \hat j}{3}-\frac{a \hat i}{i})(K. \frac{|Q| |q|}{d^2} \hat r)=
(-K. \frac{|Q||q|a}{2d^3} \hat i)+(K. \frac{|Q||q|h}{3d^3} \hat j)$\newline

$F_2 = 2(K. \frac{|Q| |q|}{d^2})\hat j$\newline

$F_3 = 3(K. \frac{|Q| |q|}{d^2})\hat w \rightarrow \hat w=(\frac{a \hat i}{2}+\frac{h \hat j}{3}) $\newline

$(3K. \frac{|Q||q|a}{2d^3} \hat i)+(3K. \frac{|Q||q|h}{3d^3} \hat j)$\newline

No eixo X\newline

$(-K. \frac{|Q||q|a}{2d^3}) + (3K. \frac{|Q||q|a}{2d^3}) = (2K. \frac{|Q||q|a}{2d^3}) \rightarrow (2K. \frac{|Q||q|a}{d^3})$\newline

$d= \frac{2h}{3}$\newline

$h=\frac{\sqrt{3}a}{2}$\newline

$\frac{2}{3}.\frac{\sqrt{3}}{2}=\frac{\sqrtr{3}a}{3}$\newline

subistituindo:\newline

$K. \frac{|Q||q|a}{(\frac{\sqrtr{3}a}{3})^3} \rightarrow  K. \frac{|Q||q|a3^3}{3\sqrt{3}a^3} \rightarrow K. \frac{\sqrt{3}|Q||q|9}{a^2}$\newline

No eixo Y\newline

$(-2K. \frac{|Q||q|}{d^2}) + (K. \frac{|Q||q|h}{3d^3}) + (3K. \frac{|Q||q|h}{3d^3})$\newline

$(-2K. \frac{|Q||q|3^2}{3a^3}) + (4K. \frac{|Q||q|}{2}.\frac{h}{d^3})$\newline

$\frac{h}{d^3}=\frac{\sqrtr{3}}{2}.a.\frac{3^3}{3\sqrt{3}a^3}=\frac{9a}{2a^2}$\newline

$(-6K. \frac{|Q||q|}{a^2})+(\frac{4k|Q||q|.9}{3.2.a^2})$\newline

$(-6K. \frac{|Q||q|}{a^2})+(6k. \frac{|Q||q|}{a^2})=0$\newline

em módulo 

$9K. \frac{\sqrt{3}|Q||q|}{a^2} \hat{i}$\newline

\textbf{2.7}\newline

$\lambda$ = densidade linear do fio\newline
$dQ = \lambda. dl = \lambda. a . d\theta $\newline
Coordenadas polares $\newline$
${X = a.cos\theta \newline Y = a . sen\theta}$\newline

$F= \frac{1}{4\pi\epsilon_0}.\frac{q.dQ}{a}.(a.cos\theta\hat{i}+a.sen\theta\hat{j}$\newline

$F= \frac{\lambda.q.a}{4\pi\epsilon_0.a}.d\theta . a \frac{(cos\theta\hat{i}+sen\theta)\hat{j}}{a}$\newline

$F = \frac{\lambda.q}{4\pi\epsilon.a} \int_0^\pi(cos\theta\hat{i}+sen\theta\hat{j}).d\theta$\newline

$\vec F = \frac{\lambda.q}{4\pi\epsilon_0.a}. 2 \hat{j} \rightarrow \frac{\lambda.q}{2\pi\epsilon_0.a}. \frac{\pi a}{\pi a}. \hat{j}\rightarrow \frac{q.\lambda\pi . a}{2\pi\epsilon_0 . a} . \hat{j} \rightarrow \frac{q\theta}{2\pi\epsilon_0 . a} . \hat{j}$\newline



\textbf{2.8}\newline

\includegraphics[width=8cm]{le3.jpeg}\newline

$d \vec F = \frac{1}{4\ppi\epsilon_0}.\frac{q.d\theta}{r^2}. \hat{j}$\newline

Os componente $dF$ em direções paralelas cancelam-se a força elétrica resultante sobre a carga $q$ é \newline

$|dF| = |d \vec F| cos\theta$ perpendicular  ao fio\newline

Onde $cos\theta = \frac{P}{r} = \frac{P}{(P^2 + Z^2)^{\frac{1}{2}}}$\newline

$dF_1 = \frac{1}{4\pi\epsilon_0} .\frac{Q\lambda.dz}{(P^2 + Z^2)^3} . P$\newline

$Fx = \frac{1}{4\pi\epsilon_0}.q.\lambda.p \int_\frac{1}{2}^\frac{-1}{2} \frac{dz}{(P^2 +Z^2)^\frac{3}{2}} = \frac{1}{4\pi\epsilon_0}.q.\lambda.P.2. \int_0^1\frac{dz}{(P^2+Z^2)^\frac{3}{2}}$\newline

$Fx = \frac{2}{4\pi\epsilon_0}.q\lambda.P.[\frac{z}{P^2\sqrt{P^2+Z^2}}]_0^1\rightarrow\frac{2}{4\pi\epsilon_0}.\frac{q.\lambda}{P}.\frac{L}{\sqrt{P^2+L^2}}$\newline

$\frac{2}{4\pi\epsilon_0}.\frac{q.\lambda}{P}.\frac{L}{\sqrt{\frac{P^2}{L^2}+1}}$\newline

Quando l tende ao fio muito longo (l \rightarrow \infty)\newline
Tem-se\newline

$Fx = \frac{2}{4\pi\epsilon_0}.\frac{q.\lambda}{P}.\frac{q.\lambda}{2\pi\epsilon_0 .P}$\newline

\end{document}






\end{document}

